\chapter{Methodology of high-performance}
\label{chap:building}

\section{Team-wide productivity}

Motivating a team is often more challenging than motivating a single individual. Individuals within teams operate with different goals, values, beliefs, and expectations. Yet the variety of team member personalities can be a positive force if each performer contributes his or her unique capabilities when and where needed.

The first critical issue in team motivation is to be clear about the definition of ``a team.'' Nearly everyone who studies teams emphasizes that it is unnecessary to use team motivation strategies when teams are defined as any group of two or more people with similar skills who are simply working together to achieve a common goal. For a team to exist (for motivational purposes), team members must play different roles or bring different skills to the table. Those different skills must be required to achieve team goals. So a team is an interdependent group of individuals, each possessing a different set of skills but who collectively possess all of the skills required to achieve team goals.

\textbf{Foster Mutual Respect for the Expertise of All Members}

Teams on which one or more members believe that they are working with people who lack adequate skills to achieve team goals have a major motivational problem. In some cases, this belief is simply incorrect. Highly competitive people sometimes distort the real situation and develop the self-protective view that one or more people on their team are inadequate. Competitive spirit is good. But bolstering self-confidence at the expense of others is immature and destructive. Bandura describes many studies in a variety of fields where ``weak link'' doubts about team member expertise have significantly reduced team effectiveness. Even though all team members vary in their expertise levels, when individuals respect and support one another, less-able team members tend to perform significantly better and work hard over time to increase their skills. Since individual team members tend to be self-focused and so think more about their own contributions and ability, team members need to be reminded about the skills of other members. One effective way to accomplish this task is to actively attribute successes to each team member’s expertise.

\textbf{Help Weaker Members Believe That Their Effort Is Vital to Team Success}

Occasionally teams must accommodate members who are novices or who for some reason are not able to do the best job for the team. When teams can’t replace weaker members, what works best to preserve team motivation? Jackson and LePine (2003) have recent and solid evidence that when team members believe that their weakest member is merely inexperienced or has faltered for some uncontrollable reason (for example, illness, accident, or a family crisis) and can improve, they will give support provided that the person is investing effort to do so.

The biggest motivational challenge on a team is faced by the weakest member. That individual must believe that what he or she contributes to the team is vital to the team’s success and that the other members expect him or her to improve and succeed. Feedback to members who are working to improve must emphasize effort, not ability. When they make progress, it is best to attribute the progress to effort. When no progress is forthcoming, they need to be urged to ``get busy, get serious and work harder.'' Avoid attributing success or failure to ability. Belief that performance is due to ability tends to discourage hard work.

In many teams the motivational challenge is not a weak link, but instead a lack of cooperation and collaboration.

\textbf{Support a Shared Belief in the Team’s Cooperative Capabilities}

Healthy teams are made up of team players who cooperate with each other. One uncooperative person can damage the motivation of even the most capable team. The obvious example is the arrogant, self-focused prima donna who invests most of his or her effort trying to look good with managers and clients—at the expense of the team. Less obvious but equally destructive is the outwardly supportive but silently devious back-stabber, whose primarily goal is to make his or her own work highly visible.

\textbf{Hold Individual Members Accountable for Contributions to the Team Effort}

One of the first team motivation studies, performed just after the turn of the century, established the principle that has been called ``social loafing.'' When people pulled as hard as possible against a rope connected to a strain gage, their best effort was recorded. When another person was added to the rope and two people pulled together, each person invested less effort in a collaborative effort than he or she did when alone. As more people were added to the rope, each person pulled less forcefully. When interviewed, most people seem unaware that they are not working as hard in a group situation as they did when alone.

\textbf{Direct the Team’s Competitive Spirit Outside the Team and the Organization}

Competition can be highly motivating for individuals or teams. Salespeople seem to thrive on it, and many people who are raised in Western cultural traditions seem to like a bit of it. One of the most common motivational team-build- ing exercises favored by organizational consultants is a field experience where teams compete with other teams to bond and build team spirit. These events are scheduled off site and are ideally held in unfamiliar settings to interrupt habitual patterns formed at work for relating to others. Teams are challenged to do something highly novel, such as build structures or navigate difficult terrain to reach a tar- get sooner or more effectively than other teams. Individuals are asked to notice how hard they are working, how much they are collaborating, and whether they have a real desire to ``win.''

Teams are defined as collections of individuals with different skill sets working together to achieve goals that require members to collaboratively apply their different skills. Collections of individuals with similar skills who tackle problems do not require team motivation strategies. In addition to motivational strategies that work with individuals, interdependent teams are most motivated when they trust both the expertise and collaborativeness of other team members as well as the determination of weaker members on Direct the Team’s Competitive Spirit Outside the Team and the Organisation
Competition can be highly motivating for individuals or teams. Salespeople seem to thrive on it, and many people who are raised in Western cultural traditions seem to like a bit of it. One of the most common motivational team-building exercises favoured by organisational consultants is a field experience where teams compete with other teams to bond and build team spirit. These events are scheduled off site and are ideally held in unfamiliar settings to interrupt habitual patterns formed at work for relating to others. Teams are challenged to do something highly novel, such as build structures or navigate difficult terrain to reach a tar- get sooner or more effectively than other teams. Individuals are asked to notice how hard they are working, how much they are collaborating, and whether they have a real desire to ``win.''

In general, team-building exercises have been found to be very effective, but they also have a potentially ugly, unintended side effect. Druckman and Bjork (1994) reviewed all studies of team building for the US National Academy of Sciences. The variety of team-building methods shared the common goal of attempting to get members of work teams to bond, collaborate, and work efficiently toward common goals by competing with other teams. The researchers concluded that many different approaches worked, but they were surprised to find that after team-building exercises, a significant number of teams were competing in a nearly suicidal fashion with other teams in their own organisation. Stories include misguided team members who were found to be modifying or deleting the electronic files, intentionally ``misplacing'' or rerouting team resources, and spreading negative rumors about members of other teams in their organizations. Apparently, fostering constant, intense rivalry can help when it is directed at the organization’s competition, but it can also engender a destructive level of internal competition and focus attention and energy away from organizational goals. The obvious motivational issue in this situation is to make certain that team-building exercises focus the team’s competitive energy on competing organizations -- not on other teams within the same organization.

\subsubsection{Summary}

Teams are defined as collections of individuals with different skill sets working together to achieve goals that require members to collaboratively apply their different skills. Collections of individuals with similar skills who tackle problems do not require team motivation strategies. In addition to motivational strategies that work with individuals, interdependent teams are most motivated when they trust both the expertise and collaborativeness of other team members as well as the determination of weaker members on their team to invest maximum effort to build their expertise. In addition, team members must believe that their own contributions to the team effort are being constantly and fairly evaluated along with the performance of the entire team. Finally, team competitiveness must be focused on opposing organizations that are struggling for the same customer base, not on teams in their own organization.