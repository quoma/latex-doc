\chapter*{Conclusion}
\addcontentsline{toc}{chapter}{Conclusion}

The objective for this dissertation was to explore modern project management methodologies and gamification theory in order to understand elements affecting performance of project participants and provide guidelines in order to increase their effectiveness.

According to a study, conducted by Dean Spitzer \cite{spitzer}, as many as 50 percent of workers said they only put enough effort into their work to hold onto their jobs. And 84 percent said they could work better -- if they wanted to. This indicates a problem of ineffectiveness of current productivity solutions.

Research of modern project management techniques led to the common feature discovery: all of them emerged as a trial to reduce complexity of a system in order to determine the cause and effect relations between its elements. That is focus on feedback.

The study of motivation and psychological mechanisms behind it sets a solid foundation to build effective tools and organisational structures.

System thinking approach allows to understand the relations between project management methodologies and personas involved into a project including their motivation systems. Prototype a mobile productivity application, that uses the notion of ``flow'' and effective feedback mechanisms was built.