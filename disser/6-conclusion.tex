\chapter*{Conclusion}
\addcontentsline{toc}{chapter}{Conclusion}

The objective of this dissertation was to investigate the problem of antimicrobial resistance and its threat to the sustainability of human race, study the development of antibiotic resistance in the hospital environment and analyze the role of different conditions and circumstances in this process.

During the research, we have studied the classical approaches in system dynamics modeling of epidemiology and understood the involved differential equations and their meaning. We have examined the suggested mathematical model, which observed a population of individuals inside a hospital and the development of antibiotic resistance in the observed infectious bacteria.

The model was implemented using a computer simulation software AnyLogic, which provided a way for observing the state of each population group under different circumstances. As it turns out, the viability of resistant bacteria strain depends on the so-called basic reproductive rate -- a parameter that showed us different scenarios of the resistant population survival.

The model described in this dissertation does not pretend to fit into some actual data from a real-life hospital. However, the observations made upon this model can allow us to understand the key parameters that are involved in this complex process, to see which variables and properties are the most sensitive for the spread of antibiotic resistance.

We hope that this work will make its contribution for solving one of the largest healthcare problems of our world nowadays and will lay down some fundamentals for further research and even more successful results.