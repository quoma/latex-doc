\section{Model and the External Population}

The model described until this point observed only the process inside the closed population of a hospital. In reality, it is almost impossible to imagine a self-sustaining hospital that does not have any connection to the external world. People are going inside and outside of any real-life hospital, therefore the mathematical model should also account for that. The individuals going in and out of the hospital may not only be patients, but also the medical personnel.

The first element to be created is the stock representing the external world. Since the external society is not in our interest, we can just simplify it to a single stock. We call it Society. Any flow going outside of the hospital will head to the Society stock and any flow coming to the system from outside will come from this stock, too. The value of the Society will change during simulation, so we will initialize it using a TotalPopulation parameter.

\begin{figure}[H]
  \centering
  \includegraphics[height=0.4\textwidth]{img/screens/society/society1}
  \caption{The Society stock and its properties}
\end{figure}

The parameters that we were using in previous chapters will help to define the initial value for Society, too. TotalPopulation is a parameter, which will have a symbolic value of 100000. This is an approximation of a small town population.

\begin{figure}[H]
  \centering
  \includegraphics[height=0.3\textwidth]{img/screens/society/society2}
  \caption{The setting of TotalPopulation parameter}
\end{figure}

In order to use the TotalPopulation parameter in the Society stock value, we need to draw some connection between them. This is the way we used to bind parameters to other objects, like stocks.

\begin{figure}[H]
  \centering
  \includegraphics[height=0.5\textwidth]{img/screens/society/society3}
  \caption{Linking the Society and TotalPopulation in the model}
\end{figure}

The next step is to create the first flow from Society stock. EnterS flow will go from Society to Susceptible, indicating the incoming individuals colonized by susceptible strain of the bacteria.

\begin{figure}[H]
  \centering
  \includegraphics[height=0.5\textwidth]{img/screens/society/society7}
  \caption{The flow of EnterS}
\end{figure}

The value of the EnterS flow will depend on two more parameters that we are going to create next. As it is seen from the corresponding figure, we are setting the value of EnterS to $\mu m$.

\begin{figure}[H]
  \centering
  \includegraphics[height=0.3\textwidth]{img/screens/society/society16}
  \caption{The properties of EnterS flow}
\end{figure}

In order to indicate the number of individuals entering the hospital from Society into different categories, we need to create a parameter $m$. This variable is a fraction of incoming people that are colonized with susceptible bacteria. The value of it may differ from one disease to another, but we set a symbolic value of 0.3 to it.

\begin{figure}[H]
  \centering
  \includegraphics[height=0.3\textwidth]{img/screens/society/society4}
  \caption{The properties of parameter $m$}
\end{figure}

Another parameter that is necessary for the flows of people is $\mu$. This variable indicates the total number of individuals entering the hospital daily. This number also provides the average length of stay in the hospital, which is equal to $1/\mu$. We set the value of this parameter to 0.1.

\begin{figure}[H]
  \centering
  \includegraphics[height=0.3\textwidth]{img/screens/society/society6}
  \caption{The parameter $\mu$ and its properties}
\end{figure}

After creating the mentioned parameters, we can add them to our model. The amount of different interactive elements of the model is increasing, so it is a good practice to keep things separated. We put the parameters near the other ones.

\begin{figure}[H]
  \centering
  \includegraphics[height=0.5\textwidth]{img/screens/society/society5}
  \caption{Placing $m$ and $\mu$ in the model}
\end{figure}

Since both parameters that we created are used in the value expression of the flow of incoming individuals infected with susceptible bacteria, we will need to connect the parameters with the flow element.

\begin{figure}[H]
  \centering
  \includegraphics[height=0.5\textwidth]{img/screens/society/society8}
  \caption{Binding $m$ and $\mu$ to the EnterS flow}
\end{figure}

