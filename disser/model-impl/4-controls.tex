\section{User Interface Controls}

In this section we are going to add some supplementary controls to make the process of simulation more convenient. AnyLogic allows users to create sliders for changing the values of any parameters.

We have selected a list of parameters that are considered to be particularly important and whose values should be able to be changed during the simulation. These parameters can be seen in the following figure.

\begin{figure}[H]
  \centering
  \includegraphics[height=0.4\textwidth]{img/screens/sliders/sliders1}
  \caption{Sensitive parameters of the model}
\end{figure}

A slider element can be created by using the user interface of the AnyLogic application. We create a slider for $\tau_1$ parameter and put it near the parameter element. The slider is connected with the variable.

\begin{figure}[H]
  \centering
  \includegraphics[height=0.4\textwidth]{img/screens/sliders/sliders2}
  \caption{Creating a slider for $\tau_1$}
\end{figure}

If we look in the properties window, we will find out that sliders have their own custom properties. There is the name of the parameter that the slider is linked to, the minimum and maximum possible values.

\begin{figure}[H]
  \centering
  \includegraphics[height=0.3\textwidth]{img/screens/sliders/sliders7}
  \caption{Properties of $\tau_1$ slider}
\end{figure}

For all parameters selected by us as being particularly important, we create the sliders exactly the same way that we did with the first one. Here, each parameter will have the corresponding slider near the located variable.

\begin{figure}[H]
  \centering
  \includegraphics[height=0.5\textwidth]{img/screens/sliders/sliders3}
  \caption{Created sliders for all sensitive parameters}
\end{figure}

\begin{figure}[H]
  \centering
  \includegraphics[height=0.3\textwidth]{img/screens/sliders/sliders6}
  \caption{Properties of $\beta$ slider}
\end{figure}

\begin{figure}[H]
  \centering
  \includegraphics[height=0.3\textwidth]{img/screens/sliders/sliders8}
  \caption{Properties of $c$ slider}
\end{figure}

\begin{figure}[H]
  \centering
  \includegraphics[height=0.3\textwidth]{img/screens/sliders/sliders9}
  \caption{Properties of $\gamma$ slider}
\end{figure}

\begin{figure}[H]
  \centering
  \includegraphics[height=0.2\textwidth]{img/screens/sliders/sliders5}
  \caption{Updating the slider value during the model execution}
\end{figure}


