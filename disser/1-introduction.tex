\chapter*{Introduction}

\addcontentsline{toc}{chapter}{Introduction}

\section{Antimicrobial Resistance Disaster}

Antimicrobial resistance is an increasing problem among the whole humanity. A solution for it still has not been discovered yet, therefore the optimal usage of the antibiotics is very important for our population. Antibiotics are very special type of medicine, which is used against bacterial diseases and often can be considered as the only possible treatment for a patient. The chemical structure of antibiotics let them effectively devastate exponentially growing populations of bacteria. In order to treat a disease effectively, doctors prescribe a specific antibiotic with specific course of usage. The antibiotics are usually needed to be used in specific doses within specific periods of time. The exact timing of treatment and accurate dosing of the medicine is very important for the right effect of the antibiotic. When patients do not adhere the prescribed conditions and rules of antibiotic usage, they allow the bacteria to develop immunity against the antibiotic medicine. This is in fact an amazing biological property, which shows how living creatures are able to adapt to different murderous enemies and conditions.

One of the main reasons why antibiotics lose their effectiveness lies behind the failure to appropriately conduct the treatment course with an antibiotic. The schedule of antibiotic usage during the treatment course has been accurately developed by pharmaceutical specialists. The course is usually designed so that the population of bacteria is effectively reduced with the medicine. The doses of the antibiotic are not too much to potentially harm the patient's organism and are not too little to result in adaption of the bacteria. Adhering to the doctor's prescriptions in terms of timing and dosing is crucial for an effective and successful treatment of a disease. Individuals that do not adhere these rules and violate the prescription conditions are not only making their own treatment ineffective and even dangerous. They are also responsible for the process of antibiotic resistance. By exposing bacteria to insufficient amounts of the antibiotic, a patient stimulates the natural ability of the organisms to develop some immunity mechanisms against the killing medicine. This way, irresponsible patients are strengthening different dangerous microorganisms and reducing the power of the weapons that are used to fight against them.

Another major problem with the protection of antibiotics is related to wrong prescriptions. In many countries, where the level of education is not as high as necessary, there are many unqualified or low-qualification medical workers. This is not only a property of developing countries, but also may appear in more developed societies, too. There are doctors who prescribe antibiotics without knowing the real reason behind the disease. In this case, the antibiotics are useless, and only get exposed to the internal micro-flora of the patient, making unpredictable changes on the bacterial level. There may occur cases when patients really need antibiotics, however doctors prescribe a wrong one. Though this may accidentally lead to treatment, the bacteria, which get exposed to the antibiotic, may start becoming resistant to that medicine. Finally, there are cases when doctors prescribe wrong doses and wrong periods of antibiotic treatment course. This is also a very probable opportunity for a bacterial population to get resistant by confronting small doses of the antibiotic.

\section{Antibiotics Abuse}

Recently, there has been a tendency among people related to farming to use antibiotics as a nutrition for animals. The feeding of animals with antibiotics tends to result in larger body sizes, meaning more meat for production. Many farmers are feeding animals like cows, horses, pigs, chicken with antibiotics and are getting fatter livestock and correspondingly larger amounts of produced meat. Eventually, this meat is eaten by consumers, i.e. by ordinary people, who in turn receive those antibiotics with the meat. This way, antibiotics can reach our organisms through the animal food that we consume. Since this kind of antibiotic "use" is not at all controlled by medical specialists, the bacteria inside our bodies again have a great opportunity to evolve and become stronger against specific chemicals. As a result, the usage of antibiotics in farming is a very destructive and seamless way of antibiotic resistance development.

As we can see, the effectiveness of existing antibiotics is rapidly decreasing as their abuse is continued to be done in different ways. The humanity could probably continue to abuse them this way, provided that the discovery of new antibiotics is being made at a sufficient rate. However, the situation in new antibiotics discovery is even worse. There hasn't been any new antibiotic discoveries since late 1980-s. This is a huge gap in research and development of antibiotics. One of the most recent discovered antibiotics is teixobactin, which is dated to January 2016. Creation of new antibiotics is a very long and difficult process that includes multiple laboratory experiments, in chemical tubes as well as on living organisms. In the light of this difficulty, it is very ineffective that people are losing the effectiveness of their only weapons against killing bacteria.

The prevention of antibiotic / antimicrobial resistance is a global process. This can't be done by a single person or organization, but rather by the collaboration of many institutions and organizations of the world. It is important to understand, how can we and should we handle this global problem and find a solution for the better future. There are many complexities involved in antimicrobial resistance. Different bacteria may be resistant or susceptible to different types of antibiotics, can infect various human populations and etc. In order to understand the problem better, we are considering the mathematical model behind this complex process. The mathematical model can be used to predict ongoing changes in the process as well as understand the key variables involved in the process. Mathematical models can be computed using computer simulations, in order to produce more detailed outcomes.

\section{Epidemiology Relevance}

Preserving and improving the health of the human population has always been a vital socio – economic issue for all over the world. In order to meet the health needs of citizens and provide public health services, the health care sector of each country monitors and controls the various operations starting from hospitalizations, patient treatments and research to vaccines and pharmaceutics. The quality of the nation’s health system is basically measured by the management of those processes, by how well they are organized and how fast they are performed \cite{per1}. It is important to have our health segment functioning timely and operatively because it costs not only financial expenditures but people’s lives as well. The World Health Organization (WHO), in 2000, ranked Kazakhstani health care system as the 64th in overall performance, and 135th by overall level of health (among 191 member nations included in the study) \cite{per1}.

The main concern of the health system and of the whole medicine as well is infectious diseases as they cause fatal consequences and their threat has been increasing \cite{per1}. This is because these diseases are now spreading geographically much faster than at any time in history as a result of the highly mobile, interdependent, and interconnected society \cite{per2}.

Infectious diseases also known as communicable or transmissible diseases are the diseases that are caused by pathogenic microorganisms, such as bacteria, viruses, parasites or fungi; the diseases can be spread, directly or indirectly, from one person to another \cite{per3}. Infectious diseases can be classified into two categories: contagious and non-contagious. Contagious diseases are easily transmitted by physical contact (hence the name-origin) with the person suffering the disease, or by their secretions or objects touched by them. \cite{per3}The most dangerous contagious disease that can cause fatal consequences and whose behavior is difficult to predict is Influenza \cite{per3}. Seasonal influenza is an acute viral infection caused by an influenza virus \cite{per3}. There are three types of seasonal influenza – A, B and C \cite{per1}. Type A influenza viruses are further typed into subtypes according to different kinds and combinations of virus surface proteins. Among many subtypes of influenza A viruses, currently influenza A (H1N1) and A(H3N2) subtypes are circulating among humans. Influenza viruses circulate in every part of the world \cite{per1}. Type C influenza cases occur much less frequently than A and B \cite{per1}. That is why only influenza A and B viruses are included in seasonal influenza vaccines \cite{per1}. Seasonal influenza is characterized by a sudden onset of high fever, cough (usually dry), headache, muscle and joint pain, severe malaise (feeling unwell), sore throat and runny nose \cite{per3}. Most people recover from fever and other symptoms within a week without requiring medical attention \cite{per3}. But influenza can cause severe illness or death in elder people and children \cite{per3}. The time from infection to illness, known as the incubation period, is about two days \cite{per3}.

Influenza worldwide produces 3–5 million severe illnesses annually and kills an estimated 250,000–500,000 people \cite{per2}. Once the flu outbreaks, it creates the epidemic and it’s hard to control its spread and prevent the morbidity and mortality. Moreover, there is a possibility each year that seasonal flu epidemic can give rise to so-called influenza pandemic. An influenza pandemic is an epidemic of an influenza virus that spreads on a worldwide scale and infects a large proportion of the human population.\cite{per3} Influenza pandemics occur when a new strain of the influenza virus is transmitted to humans from another animal species mostly pigs, chickens and ducks \cite{per3}. So, the flu virus mutates and new form of the flu is generated which cannot be treated neither by the human immunity nor by vaccinations as it’s completely novel and unknown disease. Thus, such flu spreads extremely rapidly and infects very large numbers of people. In contrast to the regular seasonal epidemics of influenza, these pandemics occur irregularly, with the 1918 Spanish flu the most serious pandemic in recent history \cite{per3}. Pandemics can cause high levels of mortality, with the 1918 Spanish influenza pandemic estimated as being responsible for the deaths of approximately 50 million people or more \cite{per3}. There have been about three influenza pandemics in each century for the last 300 years \cite{per4}. The most recent one was the 2009 flu pandemic \cite{per4}. In order to be ready for any outcome and prevent the deaths, health care segment must react as quickly as possible not to let the disease spread to other regions and kill lives. Furthermore, the flu epidemics occur mostly in Asian developing countries where there are poor living conditions and thus, there is a high risk that it can transform to pandemic \cite{per4}. And as it’s known, Kazakhstan is located near those susceptible countries like China, Malaysia and can be one of the first infected regions. So, preventive measures would be a key solution to flu tracking.

\section{Control of Epidemic Situation}

The most effective way to prevent the disease or severe outcomes from the illness is vaccination \cite{per1}. Safe and effective vaccines have been available and used for more than 60 years \cite{per1}. Among healthy adults, influenza vaccine can prevent 70\% to 90\% of influenza-specific illness \cite{per1}. Among the elderly, the vaccine reduces severe illnesses and complications by up to 60\% and deaths by 80\% \cite{per1}. Vaccination is especially important for people at higher risk of serious influenza complications, and for people who live with or care for high risk individuals.

However the vaccination process needs to be carefully considered and optimally planned. Furthermore, vaccination does not always solve the problem. What if new form of disease occurs? When the new flu outbreak happens, immediate measures and actions must be taken in order to prevent deaths. Predicting the dynamics of the spread of the disease allows developing and applying appropriate counter-measures, and can provide the optimal use of material and human resources.  For this, we should know how the disease can spread, how it can behave, who is exposed to disease and where are the hot spots of flu outbreaks. This can be done only by making predictions and prognosis based on the statistical data. And how such prediction can be achieved? One way is to mathematically calculate using formulas, regression analysis and extrapolation. However, in this case all parameters and variables are considered as uniformly distributed. For example, we can’t take into account the fact that population can change and new births and death can occur. We just use fixed and unchanging properties. So, we simply make the experiments, tests and consider possible outcomes. Obviously, such quantitative prediction will not give the estimation of the real phenomena. It means that we should also use qualitative prediction together with quantitative one. And as it’s known, qualitative prediction of disease spreading is achievable only on the basis of mathematical models. The mathematical models are discussed in more detail in the next chapter.