\begin{abstract}
	Master dissertation consists of \pageref{LastPage} pages of the typewritten text, 79 figures and 15 reference materials. The following keywords and terms are used in the paper:

	SYSTEM DYNAMICS, ANTIMICROBIAL RESISTANCE, ANTIBIOTIC RESISTANCE, EPIDEMIOLOGY, ANYLOGIC, SIMULATION, MATHEMATICAL MODEL.

	Dissertation contains an overview of the antimicrobial resistance problem, its causes and consequences. It suggests a mathematical model for simulation of the antimicrobial resistance in hospitals.

    The objective of the dissertation was to analyze the process of antimicrobial resistance development in a hospital environment in order to understand how the resistant bacteria strain gets spreaded from a group of individuals inside a hospital to the external society.

    The result of the research is a mathematical model that was consequently transferred to a computer simulation environment and executed under different parameter settings. Some important properties of the model were discussed and analyzed in the latter chapters.

\end{abstract}

\begin{otherlanguage}{russian}
\begin{abstract}
Магистерская диссертация содежрит \pageref{LastPage} страниц, 22 иллюстрации, 2 таблицы, список использованных источников -- 13 наименований. Использованы следующие ключевые слова:

ИГРОФИКАЦИЯ, ИГРОВЫЕ МЕХАНИКИ, ПОТОК, ГИБКАЯ МЕТОДОЛОГИЯ РАЗРАБОТКИ, SCRUM, RAD, TDD, FDD, БЕРЕЖЛИВАЯ РАЗРАБОТКА ПРОГРАМНОГО ОБЕСПЕЧЕНИЯ, PERT, KPI.

	Диссертация содержит обзор современных итеративных методологий разработки, индивидуальных и командных стратегий мотивации, методов игрофикации и мобильных приложений повышения эффективности.

	Целью диссертации является исследование современных методологий проектного управления и использование теории игрофикации для понимания элементов, влияющих на производительность участников проекта, а также предоставление методологий улучшения эффективности участников проекта.

	Результатом исследования является методология, основанная на причинах возникновения итеративных методологий, управления мотивацией и идеи ``потока''. Также создание прототипа приложения, которое использует понятие ``обратная связь'' для визуализации ограниченности ресурсов, таких как время и внимание.
\end{abstract}
\end{otherlanguage}