\begin{abstract}
	Master dissertation consists of \pageref{LastPage} of the typewritten text, 22 figures, 2 tables and 13 reference materials. The following keywords and terms are used in the paper:

	GAMIFICATION, GAME MECHANICS, FLOW, AGILE, SCRUM, RAPID APPLICATION DEVELOPMENT, TEST-DRIVEN DEVELOPMENT, FEATURE DRIVEN DEVELOPMENT, LEAN, PROGRAM EVALUATION AND REVIEW TECHNIQUE, KEY PERFORMANCE INDICATOR

	Dissertation contains an overview of modern iterative project management techniques, team and personal motivation strategies, gamification methods and productivity mobile applications.

	The objective was to explore modern project management methodologies and gamification theory in order to understand elements affecting performance of project participants and provide guidelines in order to increase their effectiveness.

	The result of the research is a methodology, based on  causes of emerging iterative methodologies, motivation management and the concept of ``Flow'' and an application prototype, that uses concept of ``Feedback'' in order to visualise limited resources, such as time and focus.

\end{abstract}

\begin{otherlanguage}{russian}
\begin{abstract}
Магистерская диссертация содежрит \pageref{LastPage} страниц, 22 иллюстрации, 2 таблицы, список использованных источников -- 13 наименований. Использованы следующие ключевые слова:

ИГРОФИКАЦИЯ, ИГРОВЫЕ МЕХАНИКИ, ПОТОК, ГИБКАЯ МЕТОДОЛОГИЯ РАЗРАБОТКИ, SCRUM, RAD, TDD, FDD, БЕРЕЖЛИВАЯ РАЗРАБОТКА ПРОГРАМНОГО ОБЕСПЕЧЕНИЯ, PERT, KPI.

	Диссертация содержит обзор современных итеративных методологий разработки, индивидуальных и командных стратегий мотивации, методов игрофикации и мобильных приложений повышения эффективности.

	Целью диссертации является исследование современных методологий проектного управления и использование теории игрофикации для понимания элементов, влияющих на производительность участников проекта, а также предоставление методологий улучшения эффективности участников проекта.

	Результатом исследования является методология, основанная на причинах возникновения итеративных методологий, управления мотивацией и идеи ``потока''. Также создание прототипа приложения, которое использует понятие ``обратная связь'' для визуализации ограниченности ресурсов, таких как время и внимание.
\end{abstract}
\end{otherlanguage}