\begin{abstract}
	Master dissertation consists of \pageref{LastPage} pages of the typewritten text, 69 figures and 15 reference materials. The following keywords and terms are used in the paper:

	SYSTEM DYNAMICS, ANTIMICROBIAL RESISTANCE, ANTIBIOTIC RESISTANCE, EPIDEMIOLOGY, ANYLOGIC, SIMULATION, MATHEMATICAL MODEL.

	Dissertation contains an overview of the antimicrobial resistance problem, its causes and consequences. It suggests a mathematical model for simulation of the antimicrobial resistance in hospitals.

    The objective of the dissertation was to analyze the process of antimicrobial resistance development in a hospital environment in order to understand how the resistant bacteria strain gets spreaded from a group of individuals inside a hospital to the external society.

    The result of the research is a mathematical model that was consequently transferred to a computer simulation environment and executed under different parameter settings. Some important properties of the model were discussed and analyzed in the latter chapters.

\end{abstract}

\begin{otherlanguage}{russian}
\begin{abstract}
Магистерская диссертация содержит \pageref{LastPage} страниц, 69 иллюстраций, список использованных источников -- 15 наименований. Использованы следующие ключевые слова:

СИСТЕМНАЯ ДИНАМИКА, УСТОЙЧИВОСТЬ К ПРОТИВОМИКРОБНЫМ ПРЕПАРАТАМ, УСТОЙЧИВОСТЬ К АНТИБИОТИКАМ, ЭПИДЕМИОЛОГИЯ, ANYLOGIC, СИМУЛЯЦИЯ, МАТЕМАТИЧЕСКАЯ МОДЕЛЬ.

Диссертация содержит обзор проблемы устойчивости к противомикробным препаратам, её причин и последствий. В ней предлагается математическая модель для симуляции развития устойчивости бактерий к противомикробным препаратам в больницах.

Целью диссертации является анализ процесса развития устойчивости к противомикробным препаратам в больничной среде для изучения распространения устойчивых бактерий из больниц во внешнее общество.

Результатом исследования является математическая модель, которая была преобразована в модель для симуляции в компьютерной среде и эмулирована с различными настройками параметров. Некоторые свойства модели были описаны и анализированы в последующих главах.

\end{abstract}
\end{otherlanguage}

\begin{otherlanguage}{russian}
\renewcommand{\abstractname}{Аңдатпа}
\begin{abstract}

Магистрлік диссертация \pageref{LastPage} беттен, 69 суреттен тұрады. Диссертацияда 15 дереккөзден мәлімет қолданылған. Келесі түйінді сөздер қолданылған:

ЖҮЙЕЛІК ДИНАМИКА, МИКРОБҚА ҚАРСЫ ПРЕПАРАТТАРҒА ТҰРАҚТЫЛЫҚ, АНТИБИОТИКТЕРГЕ ҚАРСЫ ТҰРАҚТЫЛЫҚ, ЭПИДЕМИОЛОГИЯ, ANYLOGIC, СИМУЛЯЦИЯ, МАТЕМАТИКАЛЫҚ МОДЕЛЬ.

Диссертацияда антибиотиктерге қарсы тұрақтылық мәселесі, оның себептері мен нәтижелері қарастырылған. Ауруханалардағы антибиотиктерге қарсы тұрақтылықтың дамуының эмуляциясы үшін математикалық модель ұсынылған.

Диссертацияның мақсаты -- ауруханалық ортадағы антибиотиктерге қарсы тұрақтылықтың дамуының барысының және тұрақты бактериялардың ауруханалардан сыртқы қоғамға таралуының зерттеуі.

Зерттеудің нәтижесінде математикалық модель ұсынылып, компьютерлық эмуляцияға арналған модельге айналдырылып, параметрларының әр-түрлі мәндерімен жүргізілген. Диссертацияның соңғы тарауларында модельдің кейбір ерекшеліктері қарастырылып талданған.

\end{abstract}
\end{otherlanguage}