\chapter{Results}

\addcontentsline{toc}{chapter}{Results}

The number of people that are colonized with sensitive bacteria is always non-zero, i.e. such people are always present in the system. This happens because such individuals are constantly entering the hospital. The population that is free from bacteria colonization is also present, since this kind of people are constantly entering from outside, too, and infected people is cured. The number of people colonized with resistant bacteria may fall to zero or stay positive. If the transmission probabilities of resistant and sensitive stains are equal, then the latter case happens under the following condition:

\begin{equation}
R_0 > \tau_1/(\tau_1 - m \mu)
\end{equation}

Here, $R_0 = \beta/(\tau_2 + \mu + \gamma)$ is a special value that indicates the rate of resistant strain reproduction in an ideal case when all of the individuals entering the hospital are not colonized with bacteria at all.