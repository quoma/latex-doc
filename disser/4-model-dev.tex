\chapter{Simulation Model Implementation}

\addcontentsline{toc}{chapter}{Simulation Model Implementation}

\section{Main Entities}

This chapter is going to describe the process of implementation, i.e. transition from a mathematical model to computer simulation. First of all we are going to create the basic elements that are essential to our model.

%\begin{figure}[!ht]
%  \centering
%  \includegraphics[height=0.5\textwidth]{img/screens/basic/basic4}
%  \includegraphics[height=0.5\textwidth]{img/screens/basic/basic3}
%  \caption{A picture of a gull.}
%\end{figure}

\begin{figure}[!ht]
    \centering
    \begin{subfigure}[b]{0.3\textwidth}
        \includegraphics[width=0.5\textwidth]{img/screens/basic/basic4}
        \caption{The stocks of 3 observed groups}
    \end{subfigure}
    ~ %add desired spacing between images, e. g. ~, \quad, \qquad, \hfill etc.
      %(or a blank line to force the subfigure onto a new line)
    \begin{subfigure}[b]{0.6\textwidth}
        \includegraphics[width=\textwidth]{img/screens/basic/basic3}
        \caption{The properties of the Susceptible stock}
    \end{subfigure}
    \caption{Creation of the basic stock elements}
\end{figure}

Here we can see how every entity is treated in AnyLogic. There have been created three stocks for three different groups of the hospital population. We name every stock as Susceptible, Uncolonized and Resistant. Each stock needs to have an initial value, which indicates how many individuals were present in each group at the beginning. We call these initial values S0, X0 and R0 respectively.

\begin{figure}[!ht]
    \centering
    \begin{subfigure}[b]{0.48\textwidth}
        \includegraphics[width=\textwidth]{img/screens/basic/basic1}
        \caption{The properties of the Resistant stock}
    \end{subfigure}
    ~ %add desired spacing between images, e. g. ~, \quad, \qquad, \hfill etc.
      %(or a blank line to force the subfigure onto a new line)
    \begin{subfigure}[b]{0.48\textwidth}
        \includegraphics[width=\textwidth]{img/screens/basic/basic2}
        \caption{The properties of the Uncolonized stock}
    \end{subfigure}
    \caption{Properties of the basic stock elements}
\end{figure}