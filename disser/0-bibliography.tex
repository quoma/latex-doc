\begin{thebibliography}{99}

\bibitem{per14}
Hethcote, H. W. "The mathematics of infectious diseases." Society for Industrial and Applied Mathematics, Vol. 42, No.1, pp. 599 – 653. 2000

\bibitem{per15}
Trottier, H., Philippe, P., Deterministic modeling of infectious diseases: theory and methods, (2001)

\bibitem{lips}
Lipsitch, M., Bergstrom, C. T., Levin, B. R. (2000). The epidemiology of antibiotic resistance in hospitals: paradoxes and prescriptions. Proceedings of the National Academy of Sciences, 97(4), 1938-1943.

\bibitem{per1}
World Health Organization (WHO), URL: http://www.who.int

\bibitem{per2}
WHO (2009) Influenza (Seasonal), Fact Sheet Number 211. Available at http://www.who.int/mediacentre/factsheets/fs211/en /index.html

\bibitem{per3}
"Influenza: Viral Infections: Merck Manual Home Edition", Merck, Retrieved 15 March 2008

\bibitem{per4}
World Health Organization, Global Alert and Response (GAR), Antiviral drugs for pandemic (H1N1) 2009: definitions and use, 22 December 2009

\bibitem{per5}
Kondratyev M.  A., “ Decision support system for infectious disease spread containing” Preprints of 13th International Student Olympiad on Automatic Control, Saint-Petersburg, (March 2010), pp. 87– 90

\bibitem{per6}
Michael Y. Li, Hal L. Smith, Liancheng Wang “Global Dynamics of an SEIR Epidemic Model with Vertical Transmission”, SIAM Journal on Applied Mathematics, Vol. 62, No. 1 (May - Sep., 2001), pp. 58-69

\bibitem{per7}
Murali Haran, An introduction to models for disease dynamics, (December 2009), pp.25-26

\bibitem{per8}
Кондратьев  М.   А.,  Ивановский  Р.  И.,  Цыбалова  Л.  М.  “Применение  агентного  подхода  к  имитационному моделированию  процесса  распространения  заболевания”, Научно-технические  ведомости  СПбГПУ.«Наука  и образование» , Vol.2, No.2 (May 2010), pp.189 –195

\bibitem{per9}
Shigui Ruan, Wendi Wang , Dynamical behavior of an epidemic model with a nonlinear incidence rate, (April, 2002), pp. 3-20

\bibitem{per10}
David J. D. Earn, Pejman Rohani, Benjamin M. Bolker, Bryan T. Grenfell , A Simple Model for Complex Dynamical Transitions in Epidemics, (March, 2003), pp. 45-70

\bibitem{per11}
Brauer, F. \& Castillo-Chávez, C. Mathematical Models in Population Biology and Epidemiology.(2001)

\bibitem{per12}
Daley, D. J. \& Gani, J. Epidemic Modeling: An Introduction. NY: Cambridge University Press.(2005)

\bibitem{per16}
John D. Sterman, Business Dynamics, (2009)

\bibitem{per17}
Evans, M.; Hastings, N.; and Peacock, B., Ch. 40, Triangular Distribution in Statistical Distributions, 3rd edition, (May, 2000), pp. 187-188

\bibitem{per18}
AnyLogic User Guide.

\bibitem{per19}
Волкова О.И. Экономика предприятия. – М.: Инфра-М, 2009. – 192с.

\bibitem{per20}
ГОСТ 12.1.005-88 ССБТ. Воздух рабочей зоны. Общие санитарно-гигиенические требования. Введ. 01.01.79. М.: Издательство стандартов, 1988, 14с.

\bibitem{per22}
СниП 23-05-91. Естественное и искусственное освещение. – М.: 1991

\bibitem{per23}
СанПин 2.2.2/2/4.1340-03. Гигиенические требования  к видеодисплейным терминалам, персональным электронно-вычислительным машинам и организации работы. Введ. 14.07.96.

\bibitem{per24}
ГОСТ 12.1.019-79 ССБТ. Электробезопасность. Общие требования и номенклатура видов защиты. Введ. С 01.07.80. М.: Издательство стандартов, 1983, 18с

\bibitem{per25}
Правила устройства электроустановок (ПУЭ). М.: Атомиздат, 1986.

\bibitem{per26}
НПБ 105-03. Категории помещений по взрывопожарной и пожарной опасности.

\bibitem{per27}
официальный сайт ГЦВП РК:/ www. gcvp.kz/ru/law

\end{thebibliography}


