\chapter*{Introduction}
\addcontentsline{toc}{chapter}{Introduction}

Preserving and improving the health of the human population has always been a vital socio – economic issue for all over the world. In order to meet the health needs of citizens and provide public health services, the health care sector of each country monitors and controls the various operations starting from hospitalizations, patient treatments and research to vaccines and pharmaceutics. The quality of the nation's health system is basically measured by the management of those processes, by how well they are organized and how fast they are performed. It is important to have our health segment functioning timely and operatively because it costs not only financial expenditures but people's lives as well. Within the twenty first century, the World Health Organization has claimed that the Kazakhstani health care system is standing on the position 64 in overall list and the 135th place on the overall health. The study was conducted among 191 different countries. [1]

The main concern of the health system and of the whole medicine as well is infectious diseases as they cause fatal consequences and their threat has been increasing. This is because these diseases are now spreading geographically much faster than at any time in history as a result of the highly mobile, interdependent, and interconnected society (World Health Organization [WHO], 2007).

Infectious diseases also known as communicable or transmissible diseases are the diseases that are caused by pathogenic microorganisms, such as bacteria, viruses, parasites or fungi; the diseases can be spread, directly or indirectly, from one person to another [2]. Infectious diseases can be classified into two categories: contagious and non-contagious. Contagious diseases are easily transmitted by physical contact (hence the name-origin) with the person suffering the disease, or by their secretions or objects touched by them [1].  Non-contagious diseases, vice-versa, require specific conditions or hosts for transmission, such as mosquitoes in malaria.

And for the last years, the majority of infections have been treated by antibiotics all over the world. Since the discovery of penicillin, the first commercial antimicrobial, people started to cure any illness with antibiotics because it's fast and effective. Even when the simple flu outbreaks, people used to take the antibiotic drug instead of natural cure in order not to miss their work and daily tasks. However, today, these antibiotics have been used so extensively that some of the microbes targeted by the drugs have adapted and become resistant to these drugs. Such process is called Antimicrobial Resistance (AMR) [3]. So, antibiotic will not help in this case, and disease can lead to fatal outcome. According to the Centers for Disease Control and Prevention (CDC), at least 2 million people become infected with antimicrobial-resistant bacteria in the US every year. Around 23,000 people die as a result [3]. According to the latest update (2014) from the National Institute of Allergy and Infectious Diseases, 5-10\% of all hospital patients develop an infection, leading to about 90,000 deaths each year, up from 13,300 patient deaths in 1992 [3]. The CDC also estimate that resistance to antibiotics in the US costs around \$20 billion a year in excess health care costs, \$35 million in societal costs and over 8 million days of labor that people spend hospitalized [3]. Therefore, Antimicrobial Resistance has become a global threat to the society and necessary measures should be taken.

Predicting the dynamics of the spread of the infectious diseases and antimicrobial resistance outcome allows developing and applying appropriate counter-measures, and can provide the optimal use of material and human resources.  For this, we should know how the disease can spread, how it can behave, who is exposed to disease and where are the hot spots of various infections. This can be done only by making predictions and prognosis based on the statistical data. And how such prediction can be achieved? One way is to mathematically calculate using formulas, regression analysis and extrapolation. However, in this case all parameters and variables are considered as uniformly distributed. For example, we can't take into account the fact that population can change and new births and death can occur. We just use fixed and unchanging properties. So, we simply make the experiments, tests and consider possible outcomes. Obviously, such quantitative prediction will not give the estimation of the real phenomena. It means that we should also use qualitative prediction together with quantitative one. And as it's known, qualitative prediction of disease spreading is achievable only on the basis of mathematical models.

A mathematical model is a description of a system using mathematical concepts and language. A model may help to explain a system and to study the effects of different components, and to make predictions about behavior. The traditional models of the disease spreading make assumptions (e.g., uniformity of individuals, their continuous uniform mixing at the modeled area) that are not sufficiently accurate. Considering the great achievements in mathematics and simulation techniques, building the models that give the most realistic outcome is a fully realizable task. Mathematical models can take many forms, including dynamical systems, statistical models, differential equations, or game theoretic models [2]. Implementation of mathematical models can be done using different methods and techniques including software and programs. This paper discusses building the mathematical model of Antimicrobial Resistance process using the system dynamics approach. System dynamics is a concept of understanding the behavior of complex systems over time. It deals with different parameters affecting the system and measuring the impacts of those variables on systems. System dynamics and the programs for it will be described in more detail in further sections.

The goal of the thesis is to research, analyze and model Antimicrobial Resistance in hospitals using the system dynamics approach on the basis of differential equations.  Such model can help to forecast disease transmission and predict changes in Antimicrobial Resistance outcome rates based on statistical data so that the vaccination amount can be economically planned. Also, this model can describe the behavior of infection spreading and possible outcomes. In order to achieve the goal, following objectives must be reached:

–	analysis of mathematical models and how they can be used in various fields such as health care;

–	analysis of different approaches to modeling;

–	analysis of existing models of infectious disease spreading;

–	the choice of the modern approach to modeling the Antimicrobial Resistance in hospitals and justification of the choice;

–	analysis of model simulation systems;

–	implement the model of spread of the disease on the basis of the approach on a computer;

–	obtain the automatic prognosis of Antimicrobial Resistance;

–	analysis of the final model : calibration of the parameters of the model, check the validation of the model, evaluation of the sensitivity of the model;

To solve the objectives of the work, various research methods were used. Specifically, functional analysis, extrapolation, differential equations, probability theory and mathematical statistics: the theory of stochastic processes, correlation analysis, regression analysis and methods of the theory evaluation.

The developed model is based on the methods of simulation modeling, in particular, the system dynamics modeling, its realization – on the model simulating software called AnyLogic.

The scientific novelty of this work is as follows:

•	appliance of mathematics in health care segment was proved to be achievable task;

•	mathematical model for Antimicrobial Resistance in hospitals was made;

•	new software tool for simulating the model was used and analyzed;

•	the system dynamics approach to the construction of models of complex processes and systems are developed to create an imitation model of the disease spreading;

•	the flu spreading prognosis and prediction is made;

•	the model can be applied not only for Antimicrobial Resistance in hospitals, but also for other infectious diseases transmission;

The simulation of the model can show us how the disease spreads in hospitals and how antimicrobial resistant bacteria can occur. Created model simulation allows us to build a forecast development of the epidemic situation in the particular territory. Through the use of system dynamics approach, the proposed model can easily let take into account any measures against disease (vaccination, quarantine and etc..). The developed system allows to quantify the effectiveness of various measures and select the most appropriate, based on from the current level of disease.

The practical use of such epidemic models must rely heavily on the realism put into the models. This doesn't mean that a reasonable model can include all possible effects in the simplest possible fashion to maintain major components that influence disease propagation.

Great care should be taken before epidemic models are used for prediction of real phenomena. However, even simple models should, and often do, pose important questions about the underlying mechanisms of infection spread and possible means of control of the disease. So, this project offers the latter solution and can be implemented in the health organizations, hospitals, clinics, and ministry of Health. Also, it can be used as an application in the epidemiological centers where the various researches are carried out.
