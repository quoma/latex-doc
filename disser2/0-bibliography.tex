\begin{thebibliography}{99}

\bibitem{1}
WHO (2009) Influenza (Seasonal), Fact Sheet Number 211. Available at http://www.who.int/mediacentre/factsheets/fs211/en /index.html

\bibitem{2}
World Health Organization (WHO), URL: http://www.who.int

\bibitem{3}
Center for Disease Control and Prevention (CDC), URL: http://www.cdc.gov/drugresistance/about.html

\bibitem{4}
Health at Glance 2015, OECD Indicators Available at http://dx.doi.org/10.1787/health_glance-2015-en

\bibitem{5}
World Health Organization (WHO), Antimicrobial Resistance global report on surveillance, France, 2014

\bibitem{6}
The state of the world’s antibiotics 2015, Center for Disease Control and Prevention (CDC), Washington DC, 2015

\bibitem{7}
2011 Retail Meat Report, National  Antimicrobial resistance Monitoring System, February, 2011

\bibitem{8}
Global trends in antimicrobial use in animals and humans,T. van Boeckel, Setah pharmaceptical advisory group symposium

\bibitem{9}
David J. D. Earn, Pejman Rohani, Benjamin M. Bolker, Bryan T. Grenfell , A Simple Model for Complex Dynamical Transitions in Epidemics, (March, 2003), pp. 45-70

\bibitem{10}
Brauer, F. \& Castillo-Chávez, C. Mathematical Models in Population Biology and Epidemiology.(2001)

\bibitem{11}
Daley, D. J. \& Gani, J. Epidemic Modeling: An Introduction. NY: Cambridge University Press.(2005)

\bibitem{12}
Murali Haran, An introduction to models for disease dynamics, (December 2009), pp.25-26

\bibitem{13}
Evans, M.; Hastings, N.; and Peacock, B., Ch. 40, Triangular Distribution in Statistical Distributions, 3rd edition, (May, 2000), pp. 187-188

\bibitem{14}
Hethcote, H. W. (2000). "The mathematics of infectious diseases." Society for Industrial and Applied Mathematics, Vol. 42, No.1, pp. 599 – 653.

\bibitem{15}
Trottier, H., \& Philippe, P., Deterministic modeling of infectious diseases: theory and methods, (2001)


\end{thebibliography}


