\chapter*{Hospital Model}

\addcontentsline{toc}{chapter}{Hospital Model}

Antibiotic resistance is an increasing problem among the whole humanity. A solution for it still has not been discovered yet, therefore the optimal usage of the antibiotics is very important for our population. Antibiotics are very special type of medicine, which is used against bacterial diseases and often can be considered as the only posssible treatment for a patient. The chemical structure of antibiotics let them effectively devastate exponentially growing populations of bacteria. In order to treat a disease effectively, doctors prescribe a specific antibiotic with specific course of of usage. The antibiotics are usually need to be used specific doses within specific periods of time. The exact timing of treatment and accurate dosing of the medicine is very important for the right effect of the antibiotic. When patients do not adhere the prescribed conditions and rules of antibiotic usage, they allow the bacteria to develop an inmmunity against the antibiotic medicine. This is in fact an amazing biological property, which shows how living creatures are able to adapt to different murderous enemies and conditions.

One of the main reasons why antibiotics lose their effectiveness lies behind the filure to appropriately conduct the treatment course with an antibiotic. The schedule of antibiotic usage during the treatment course has been accurately developed by pharmaceutical specialists. The course is usually designed so that the population of bacteria is effectively reduced with the medicine. The doses of the antibiotic are not too much to potentially harm the patient's organism and are not too little to result in adaption of the bacteria. Adhering to the doctor's prescriptions in terms of timing and dosing is crucial for an effective and successful treatment of a disease. Individuals that do not adhere these rules and violate the prescription conditions are not only making their own treatment ineffective and even dangerous. They are also responsible for the process of antibiotic resistandce. By exposing bacteria to insufficient amounts of the antibiotic, a patient stimulates the natural ability of the organisms to develop some immunity mechanisms against the killing medicine. This way, irresponsible patients are strenghtening different dangerous microorganisms and reducing the power of the weapons that are used to fight against them.

Another major problem with the protection of antibiotics is related to wrong prescriptions. In many countries, where the level of education is not as high as necessary, there are many unqulified or low-qualification medical workers. This is not only a property of developting countries, but also may appear in more developed societies, too. There are doctors who prescribe antibiotics without knowing the real reason behind the disease. In this case, the antibiotics are useless, and only get exposed to the internal micro-flora of the patient, making unpredictable changes on the bacterial level. There may occure cases when patients really need antibiotics, however doctors prescribe a wrong one. Though this may accidentally lead to treatment, the bacteria, which get exposed to the antibiotic may start becoming resistant to that medicine. Finally, there are cases when doctors prescribe wrong doses and wrong periods of antibiotic treatment course. This is also a very probable opportunity for a bacterial population to get resistant by confronting small doses of the antibiotic.

Recently, there has been a tendency among people related to farming to use antibiotics as a nutriotion for animals. The feeding of animals with antiotics tends to result in larger body sizes, meaning more meat for production. Many farmers are feeding animals like cows, horses, pigs, chicken with antibiotics and are getting fatter livestock and correspondingly larger amounts of produced meat. Eventually, this meat is eaten by consumers, i.e. by ordinary people, who in turn receive those antibiotics with the meat. This way, antibiotics can reach our organisms through the animal food that we consume. Since this kind of antibiotic "use" is not at all controlled by medical specialists, the bacteria inside our bodies again have a great opportunity to evolve and become stronger against specific chemicals. As a result, the usage of antibiotics in farming is a very destructive and seamless way of antibiotic resistance development.

As we can see, the effectiveness of existing antibiotics is rapidly decreasing as their abuse is continued to be done in different ways. The humanity could probably continue to abuse them this way, provided that the discovery of new antibiotics is being made at a sufficient rate. However, the situation in new antibiotics discovery is even worse. There hasn't been any new antibiotic discoveries since late 1980s. This is a huge gap in research and development of antibiotics. One of the most recent discovered antibiotics is teixobactin, which is dated to January 2016. Creation of new antibiotics is a very long and difficult process that includes multiple labarotial experiments, in chemical tubes as well as on living organisms. In the light of this difficulty, it is very ineffective that people are losing the effetiveness of their only weapons against killing bacterias.

The prevention of antibiotic/antimicrobial resistance is a global process. This can't be done by a single person or organisation, but rather by the collaboration of many institutions and organisations of the world. It is important to understand, how can we and should we handle this global problem and find a solution for the better future. There are many complexities involved in antimicrobial resistance. Different bacteria may be resistant or susceptible to different types of antibiotics, can infect various human populations and etc. In order to understand the problem better, we are considering the mathematical model behind this complex process. The mathematical model can be used to predict ongoing changes in the process as well as understand the key variables involved in the process. Mathematical models can be computed using computer simulations, in order to produce more detailed outcomes.

There has recently been conducted a study that examined the case of transmission of resistant and susceptible bacteria in hospitals. The research of such processes is particularly important for understanding the nosocomial transmission rates of the bacteria. It is significant for the healthcare workers to be able to control the spread of resistant bacteria and reduce their antimicrobial resistance. The model may predict whether the usage of an antibiotic which does not have corresponding resistant bacteria present in the hospital will be useful to decrease the level of resistance in the examined bacteria.
