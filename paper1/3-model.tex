\chapter*{Mathematical Model Description}

\addcontentsline{toc}{chapter}{Mathematical Model Description}

We are going to observe a mathematical model proposed by \cite{lips}, which describes the epidemiology of antimicrobial resistant infections in hospitals. The model examines a population of people in a closed environment, where new patients come into the system at some rate and go out of the system respectively. The model can be applied for the diseases that are transmitted through the skin, respiratory or digestive organs. A few examples of such bacteria are Escherichia coli, Staphylococcus, Enterococcus, Klebsiella pneuoniae and others. The mentioned bacteria are dangerous enough to cause a painful or even lethal infection. The hospital environment was chosen because the transmission of these bacteria often takes place inside the hospitals. Different patients may accidentally contact with each other and spread their infections to one another, as well as to medical personnel. The medical workers are also often responsible for the spread of infections inside the hospital, since they may pass the bacteria between their patients. This might happen as a result of unsufficient hygiene with respect to their hands and inventory items. The bacteria in the system are continuously encountering the antibiotics that are used in the hospital. As a result, the patients may always get infected accidentally and the bacteria can pass their antibiotic resistant plasmids to other strains.

The model observes the transmission dynamics of the bacteria in a hospital, with regard to the usage of two different antibiotics - drug A and drug B. Some portion of all individuals that enter the hospital are already colonized with the bacteria, that we are looking at.